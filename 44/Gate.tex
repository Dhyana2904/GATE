%\iffalse
\let\negmedspace\undefined
\let\negthickspace\undefined
\documentclass[journal,12pt,twocolumn]{IEEEtran}
\usepackage{cite}
\usepackage{amsmath,amssymb,amsfonts,amsthm}
\usepackage{algorithmic}
\usepackage{graphicx}
\usepackage{textcomp}
\usepackage{xcolor}
\usepackage{txfonts}
\usepackage{listings}
\usepackage{enumitem}
\usepackage{mathtools}
\usepackage{gensymb}
\usepackage{comment}
\usepackage[breaklinks=true]{hyperref}
\usepackage{tkz-euclide} 
\usepackage{listings}
\usepackage{gvv}                                        
\def\inputGnumericTable{}                                 
\usepackage[latin1]{inputenc}                                
\usepackage{color}                                            
\usepackage{array}                                            
\usepackage{longtable}                                       
\usepackage{calc}                                             
\usepackage{multirow}                                         
\usepackage{hhline}                                           
\usepackage{ifthen}                                           
\usepackage{lscape}

\newtheorem{theorem}{Theorem}[section]
\newtheorem{problem}{Problem}
\newtheorem{proposition}{Proposition}[section]
\newtheorem{lemma}{Lemma}[section]
\newtheorem{corollary}[theorem]{Corollary}
\newtheorem{example}{Example}[section]
\newtheorem{definition}[problem]{Definition}
\newcommand{\BEQA}{\begin{eqnarray}}
\newcommand{\EEQA}{\end{eqnarray}}
\newcommand{\define}{\stackrel{\triangle}{=}}
\theoremstyle{remark}
\newtheorem{rem}{Remark}
\begin{document}
\bibliographystyle{IEEEtran}
\vspace{3cm}
\title{\textbf{IN-2023}}
\author{EE23BTECH11210-Dhyana Teja Machineni$^{*}$% <-this % stops a space
}
\maketitle
\newpage
\bigskip

\textbf{QUESTION:}\\
A continuous real-valued signal $x\brak{t}$ has finite positive energy and $x\brak{t} = 0$, $\forall$ $t < 0$. From the list given below, select ALL the signals whose
continuous-time Fourier transform is purely imaginary.\\
\begin{enumerate}
\item$x\brak{t} + x\brak{-t}$
\item$x\brak{t} - x\brak{-t}$
\item$j\brak{x\brak{t} + x\brak{-t}}$
\item$j\brak{x\brak{t} - x\brak{-t}}$
\end{enumerate}
\hfill{(GATE IN 2023)}\\
\solution\\
\begin{table}[h]
         \label{tab:table}
         \renewcommand{\arraystretch}{1.5}
\begin{tabular}{|c|c|}
\hline
Parameter & Description  \\\hline
\( x(t) \) & Continuous real valued signal  \\\hline
\(x^*(t) \) & conjugate of x(t) \\\hline
\(t \) & time \\\hline
\(\omega \) & angular velocity of the signal \\\hline
\(X(\omega)\)& Fourier Transfom of x(t)\\\hline
\(X(\omega)^*\) & Conjugate of \(X(\omega)\) \\ \hline
\end{tabular}

         \caption{Variables and their descriptions}
     \end{table}\\
Fourier transform of an real and odd signal$x\brak{t}$ is purely imaginary.\\
\begin{align}
\mathcal{F}\{x(t)\} &= X\brak{f}\\
X\brak{f}&=\int_{-\infty}^{\infty} x\brak{t} e^{-j 2\pi f t} \ dt\\
X\brak{f}^*&=\int_{-\infty}^{\infty} x\brak{t} e^{j 2\pi f t} \ dt\\
X\brak{f}^*&=\int_{-\infty}^{\infty} x\brak{-t} e^{-j 2\pi f t} \ dt\\
X\brak{f}^*&=-X\brak{f}
\end{align}
 Fourier transform of an imaginary even signal $jx\brak{t}$ is purely imaginary.\\
\begin{align}
\mathcal{F}\{x(t)\} &= X\brak{f}\\
X\brak{f}&=\int_{-\infty}^{\infty} jx\brak{t} e^{-j 2\pi f t} \ dt\\
X\brak{f}^*&=-\int_{-\infty}^{\infty} jx\brak{t} e^{j 2\pi f t} \ dt\\
X\brak{f}^*&=-\int_{-\infty}^{\infty} jx\brak{-t} e^{-j 2\pi f t} \ dt\\
X\brak{f}^*&=-X\brak{f}
\end{align}
\begin{align} 
x\brak{t} &= 
\begin{cases} 
0 & \text{for } t < 0 \\
t & \text{for } t \geq 0 
\end{cases}\\
x\brak{-t} &= 
\begin{cases} 
-t & \text{for } t \leq 0 \\
0 & \text{for } t > 0 
\end{cases}
\end{align}
1)$x\brak{t} + x\brak{-t}$
\begin{align}
  f\brak{t}&= x\brak{t} + x\brak{-t}\\
  f\brak{t} &= 
\begin{cases} 
-t & \text{for } t < 0 \\
t & \text{for } t \geq 0 
\end{cases}\\
   f\brak{-t} &= 
\begin{cases} 
t & \text{for } t > 0 \\
-t & \text{for } t \leq 0 
\end{cases}\\
  f\brak{-t}&=f\brak{t}\\
  \mathcal{F}\{f(t)\}&=\int_{-\infty}^{\infty} f\brak{t} e^{-j 2\pi f t} \ dt\\
  &=2\int_{0}^{\infty} t  cos\brak{2\pi ft}  \ dt
\end{align}
$\therefore$ Fourier Transform is not Purely imaginary.\\
2) $x\brak{t} - x\brak{-t}$
\begin{align}
  f\brak{t}&= x\brak{t} - x\brak{-t}\\
  f\brak{t} &= 
\begin{cases} 
t & \text{for } t < 0 \\
t & \text{for } t \geq 0 
\end{cases}\\
f\brak{-t} &= 
\begin{cases} 
-t & \text{for } t > 0 \\
-t & \text{for } t \leq 0 
\end{cases}\\
f\brak{t}&=-f\brak{-t}\\
  \mathcal{F}\{f(t)\}&=\int_{-\infty}^{\infty} f\brak{t} e^{-j 2\pi f t} \ dt\\
  &=2j\int_{0}^{\infty} t  sin\brak{2\pi ft}  \ dt
\end{align}
$\therefore$ Fourier Transform is purely imaginary.\\
3)$j\brak{x\brak{t} + x\brak{-t}}$
\begin{align}
  f\brak{t}&= j\brak{x\brak{t} + x\brak{-t}}\\
  f\brak{t} &= 
\begin{cases} 
-jt & \text{for } t < 0 \\
jt & \text{for } t \geq 0 
\end{cases}\\
   f\brak{-t} &= 
\begin{cases} 
jt & \text{for } t > 0 \\
-jt & \text{for } t \leq 0 
\end{cases}\\
  f\brak{-t}&=f\brak{t}\\
  \mathcal{F}\{f(t)\}&=\int_{-\infty}^{\infty} jt e^{-j 2\pi f t} \ dt\\
  &=2j\int_{0}^{\infty} t  cos\brak{2\pi ft}  \ dt
\end{align}
$\therefore$ Fourier Transform is Purely imaginary.\\
4)$j\brak{x\brak{t} - x\brak{-t}}$
\begin{align}
  f\brak{t}=j\brak{x\brak{t} - x\brak{-t}}\\
  f\brak{t} &= 
\begin{cases} 
jt & \text{for } t < 0 \\
jt & \text{for } t \geq 0 
\end{cases}\\
f\brak{-t} &= 
\begin{cases} 
-jt & \text{for } t > 0 \\
-jt & \text{for } t \leq 0 
\end{cases}\\
f\brak{t}&=-f\brak{-t}\\
  \mathcal{F}\{f(t)\}&=\int_{-\infty}^{\infty} f\brak{t} e^{-j 2\pi f t} \ dt\\
  &=-2\int_{0}^{\infty} t  sin\brak{2\pi ft}  \ dt
\end{align}
$\therefore$ Fourier Transform is not Purely imaginary.\\
\end{document}
