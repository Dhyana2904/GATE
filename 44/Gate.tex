%\iffalse
\let\negmedspace\undefined
\let\negthickspace\undefined
\documentclass[journal,12pt,twocolumn]{IEEEtran}
\usepackage{cite}
\usepackage{amsmath,amssymb,amsfonts,amsthm}
\usepackage{algorithmic}
\usepackage{graphicx}
\usepackage{textcomp}
\usepackage{xcolor}
\usepackage{txfonts}
\usepackage{listings}
\usepackage{enumitem}
\usepackage{mathtools}
\usepackage{gensymb}
\usepackage{comment}
\usepackage[breaklinks=true]{hyperref}
\usepackage{tkz-euclide} 
\usepackage{listings}
\usepackage{gvv}                                        
\def\inputGnumericTable{}                                 
\usepackage[latin1]{inputenc}                                
\usepackage{color}                                            
\usepackage{array}                                            
\usepackage{longtable}                                       
\usepackage{calc}                                             
\usepackage{multirow}                                         
\usepackage{hhline}                                           
\usepackage{ifthen}                                           
\usepackage{lscape}

\newtheorem{theorem}{Theorem}[section]
\newtheorem{problem}{Problem}
\newtheorem{proposition}{Proposition}[section]
\newtheorem{lemma}{Lemma}[section]
\newtheorem{corollary}[theorem]{Corollary}
\newtheorem{example}{Example}[section]
\newtheorem{definition}[problem]{Definition}
\newcommand{\BEQA}{\begin{eqnarray}}
\newcommand{\EEQA}{\end{eqnarray}}
\newcommand{\define}{\stackrel{\triangle}{=}}
\theoremstyle{remark}
\newtheorem{rem}{Remark}
\begin{document}
\bibliographystyle{IEEEtran}
\vspace{3cm}
\title{\textbf{IN-2023}}
\author{EE23BTECH11210-Dhyana Teja Machineni$^{*}$% <-this % stops a space
}
\maketitle
\newpage
\bigskip

\textbf{QUESTION:$44$}\\
A continuous real-valued signal $x\brak{t}$ has finite positive energy and $x\brak{t} = 0$, $\forall$ $t < 0$. From the list given below, select ALL the signals whose
continuous-time Fourier transform is purely imaginary.\\
\begin{enumerate}
\item$x\brak{t} + x\brak{-t}$
\item$x\brak{t} - x\brak{-t}$
\item$j\brak{x\brak{t} + x\brak{-t}}$
\item$j\brak{x\brak{t} - x\brak{-t}}$
\end{enumerate}
\hfill{(GATE IN 2023)}\\
\solution\\
\begin{table}[h]
         \label{tab:table}
         \renewcommand{\arraystretch}{1.5}
\begin{tabular}{|c|c|}
\hline
Parameter & Description  \\\hline
$u\brak{n}$& unit step function \\\hline
$z\brak{n}$ & convolution of $x\brak{n}$,$y\brak{-n}$ \\\hline
$Z\brak{e^{j\omega}}$ & DFT of $z\brak{n}$ \\\hline
\end{tabular}

         \caption{Variables and their descriptions}
     \end{table}\\
Fourier transform of an real and odd signal$x\brak{t}$ is purely imaginary.\\
\begin{align}
\mathcal{F}\{x(t)\} &= X\brak{\omega}\\
X\brak{\omega}&=\int_{-\infty}^{\infty} x\brak{t} e^{-j \omega t} \ dt\\
&=\int_{-\infty}^{\infty} x\brak{-t} e^{j \omega t} \ dt\\
&=-\int_{-\infty}^{\infty} x\brak{t} e^{j \omega t} \ dt\\
X\brak{\omega}^*&=\int_{-\infty}^{\infty} x\brak{t} e^{j \omega t} \ dt\\
X\brak{\omega}&=-X\brak{\omega}^*
\end{align}
 Fourier transform of an imaginary even signal $jx\brak{t}$ is purely imaginary.\\
\begin{align}
\mathcal{F}\{x(t)\} &= X\brak{\omega}\\
X\brak{\omega}&=\int_{-\infty}^{\infty} jx\brak{t} e^{-j \omega t} \ dt\\
&=\int_{-\infty}^{\infty} jx\brak{-t} e^{j \omega t} \ dt\\
&=\int_{-\infty}^{\infty} jx\brak{t} e^{j \omega t} \ dt\\
X\brak{\omega}^*&=-\int_{-\infty}^{\infty} jx\brak{t} e^{j \omega t} \ dt\\
X\brak{\omega}&=-X\brak{\omega}^*
\end{align}
1)$x\brak{t} + x\brak{-t}$\\
\begin{align}
  f\brak{t}= x\brak{t} + x\brak{-t}\\
  f\brak{-t}=x\brak{-t} + x\brak{t}\\
  f\brak{t}=f\brak{-t}
\end{align}
$\therefore$ $x\brak{t} + x\brak{-t}$ is not Purely imaginary.\\
2) $x\brak{t} - x\brak{-t}$\\
\begin{align}
  f\brak{t}= x\brak{t} - x\brak{-t}\\
  f\brak{-t}=x\brak{-t} - x\brak{t}\\
  f\brak{-t}=-f\brak{t}
\end{align}
$\therefore$ $x\brak{t} - x\brak{-t}$ is purely imaginary.\\
3)$j\brak{x\brak{t} + x\brak{-t}}$
\begin{align}
  f\brak{t}=j\brak{x\brak{t} + x\brak{-t}}\\
  f\brak{-t}=j\brak{x\brak{-t} + x\brak{t}}\\
  f\brak{t}=f\brak{-t}
\end{align}
$\therefore$ $j\brak{x\brak{t} + x\brak{-t}}$ is Purely imaginary.\\
4)$j\brak{x\brak{t} - x\brak{-t}}$
\begin{align}
   f\brak{t}=j\brak{x\brak{t} - x\brak{-t}}\\
  f\brak{-t}=j\brak{x\brak{-t} - x\brak{t}}\\
  f\brak{t}=-f\brak{-t}
\end{align}
$\therefore$ $j\brak{x\brak{t} + x\brak{-t}}$ is not Purely imaginary.\\
\end{document}
